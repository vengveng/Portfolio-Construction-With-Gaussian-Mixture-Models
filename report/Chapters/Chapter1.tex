\chapter{Introduction}
\label{chap:intro}
\lhead{Chapter 1. \emph{Introduction}}

Harry Markowitz's mean-variance approach gave us the elegant intuition for the trade-off between expected return and volatility. This lesson, however, was predicated on the assumption that asset returns are normally (or at least elliptically) distributed. Empirical studies have long demonstrated that returns exhibit skewness and heavy tails, features that the Gaussian model cannot accommodate. Ignoring these higher-order characteristics leads to systematic underestimation of downside risk and mispricing of asymmetric payoffs.

From early surveys of capital budgeting to recent evidence of skewness-seeking in hedge-fund allocations (\cite{maoSurveyCapitalBudgeting1970}, \cite{benuzziSkewnessseekingBehaviorFinancial2024}), the fact that investors care deeply about asymmetry has been thoroughly attested to. Practitioners routinely favor portfolios whose return distributions offer the potential for infrequent large gains, even at the cost of modest increases in variance. It follows that any model that aspires to capture realistic market prices must account for skewness. 

Embedding skewness directly via non-Gaussian densities quickly sacrifices the analytical clarity of the mean-variance framework. Once returns follow, for instance, a skewed Student-$t$ distribution, closed-form expressions for portfolio moments vanish and must be estimated by numerical integration or simulation. Despite offering performance benefits, optimization under these densities becomes a computationally intensive black box, obscuring the intuitive "risk-return" trade-off.

A natural alternative approach is to focus on embedding skewness into the utility function rather than the distribution itself. However, as \cite{ebertSkewnessPreferencePopularity2016} demonstrate, skewness preferences are incompatible with classical expected utility theory (EUT). This "impossibility result" states that no smooth, concave expected-utility function can display a strong preference for skewness. Intuitively, a Taylor expansion around any wealth level demonstrates that the first-order (mean) and second-order (variance) effects cancel out when you introduce an infinitesimal skew in probabilities, leaving only a third-order term. Hence, in the immediate neighborhood of any reference wealth, expected utility is effectively blind to small skew shifts: it treats a tiny chance of a large gain just the same as a tiny chance of a large loss.

One way of avoiding this limitation of EUT is to adopt behavioral utility models. Two main extensions to EUT have enabled the modeling of skewness: Rank-Dependent Expected Utility (RDEU) and Cumulative Prospect Theory (CPT) (\cite{quigginTheoryAnticipatedUtility1982}, \cite{tverskyAdvancesProspectTheory1992}). RDEU extends EUT using a Probability Weighting Function (PWF) to inflate the perceived likelihood of tail events. CPT builds on RDEU by introducing separate value and PWFs for gains versus losses. These modifications allow for skewness-seeking on the upside to differ from skewness-aversion on the downside, which RDEU cannot model (\cite{kligerTheoriesChoiceRisk2009}). 

These utility models do provide a better descriptive account of how real assets are priced. This success, however, hinges on incorporating the non-rationality of human decision making into the utility framework. Indeed, the way in which a financial decision is made and the way it should be made are two distinct questions. In other words, a prudent investor ought to assess whether a rational decision and a decision that maximizes the utility function that captures their biases are one and the same. It is thus unsurprising that applications of behavioral utility frameworks to portfolio construction show a general trend of underperformance compared to classical methods. For instance, in \cite{cuiDecisionMakingCumulative2024}, a CPT-based strategy attained considerably lower risk-return characteristics than the equally-weighted portfolio. Furthermore, as \cite{luxenbergPortfolioOptimizationCumulative2024} point out, CPT-based optimization is computationally challenging, and analytical solutions have as of yet been devised only for approaches with considerable axiomatic limitations. 

An alternative way of incorporating skewness is by means of Mean-Variance-Skewness-Kurtosis models (MVSK), initially proposed by \cite{konnoMeanVarianceSkewnessPortfolioOptimization1995}. These models have thoroughly demonstrated their ability to offer performance advantages over the traditional methods, as \cite{mandalHigherorderMomentsPortfolio2024} summarize. However, the MVSK framework relies on ad hoc parametrization of skewness (kurtosis) preference (aversion) parameters. Additionally, any single skewness metric conflates tail weight and tail distance and thus cannot, on its own, distinguish different forms of asymmetry. These shortcomings of MVSK models, coupled with the computational challenges and the lack of analytical insights, leave much to be desired in terms of interpretability.

To reconcile tractability with a faithful representation of asymmetry, we return to the density-based introduction of higher moments. Gaussian Mixture Models (GMMs) provide a natural first step toward flexible density estimation. By representing the return distribution as a weighted sum of Gaussian components, GMMs can approximate a wide array of shapes, while allowing us to preserve an "MV-style" interpretation of optimal solutions. The conceptual groundwork for the use of GMM in portfolio construction was laid by \cite{luxenbergPortfolioConstructionGaussian2022}, and later expanded upon by \cite{wangRobustContextualPortfolio2022}. Building on these, this thesis tests GMM-based portfolios across five major equity indices, and with a broader configuration space than the previous works. Our out-of-sample findings diverge from those of the second paper, for reasons we will analyze in Chapter~\ref{chap:empirical}.

More importantly, we introduce a kernel-density (KDE) variant that eliminates the computationally heavy tuning step inherent in GMM, and show that this non-parametric alternative delivers markedly superior stability. KDE "lets the data speak" by placing a smooth kernel at each observed return, automatically capturing any degree of skewness or tail-fatness without specifying functional forms. In Chapter~\ref{chap:math}, we detail the mathematical foundations of GMM and KDE. Then, in Chapter~\ref{chap:portfolio}, we derive portfolios under each estimator, establish their convexity, and interpret the resulting analytical insights. Finally, in Chapter~\ref{chap:empirical}, we present a comparative empirical analysis of these methods, provide guidance for practitioners, and discuss future research directions.
