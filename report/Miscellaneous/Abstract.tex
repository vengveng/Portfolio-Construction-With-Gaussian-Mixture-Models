\chapter*{Abstract}
\addcontentsline{toc}{chapter}{Abstract}
\begin{otherlanguage}{english}
In this paper, we review and implement the portfolio construction technique using Gaussian mixture models as described by \cite{luxenbergPortfolioConstructionGaussian2022}. To complement the original study, we conduct a preliminary empirical analysis of this approach. We also explore a related method based on kernel density estimation. The first section explains the mathematical foundations of both models. The second section presents a comparative empirical analysis, in which we narrow down optimal configurations and assess the practical viability of each approach.
\end{otherlanguage}

\vspace{3ex}

\phantomsection
\begin{otherlanguage}{french}
    Dans ce mémoire, nous examinons et implémentons la technique de construction de portefeuille basée sur les modèles de mélanges gaussiens, telle que décrite par \cite{luxenbergPortfolioConstructionGaussian2022}. Afin de compléter l'étude originale, nous réalisons une analyse empirique préliminaire de cette approche. Nous explorons également une méthode apparentée utilisant l'estimation par noyaux. La première partie expose les fondements mathématiques des deux modèles. La seconde présente une analyse empirique comparative, au sein de laquelle nous identifions les configurations optimales et évaluons la viabilité pratique de chaque approche.
\end{otherlanguage}

\vfill
% \vspace{1em}
  \textbf{Keywords:} Asset Allocation; Portfolio Choice; Semiparametric Methods; Financial Econometrics.\\
  \textbf{JEL Classification:} G11; C14; C58
